% 设置文档类为ctexart,使用12pt字体大小和A4纸张大小
\documentclass[12pt,a4paper]{ctexart}

% 导入必要的宏包
\usepackage{graphicx} % 用于插入图片
\usepackage{float} % 用于控制浮动体的位置
\usepackage{enumitem} % 用于定制列表
\usepackage[margin=2cm]{geometry} % 设置页边距为2cm

% 进一步设置页面底部边距为2cm
\geometry{bottom=2cm}

% 设置文档的标题
\title{实验报告}
\date{}

% 开始文档
\begin{document}

% 插入标题图片
\begin{figure}[H]
    \centering
    \includegraphics[width=6cm]{name.png} % 插入名为name.png的图片,宽度设置为6cm
\end{figure}

% 插入标题,使用Huge字体大小并加粗
% \vspace{0.5cm} % 在图片和标题之间插入0.5cm的垂直空间
\begin{center}
    \begin{Huge}
        \textbf{实\hspace{2cm}验\hspace{2cm}报\hspace{2cm}告}
    \end{Huge}
\end{center}

% 插入课程学期,居中显示
\begin{center}
    \textbf{(2024-2025-1)}
\end{center}

% 插入基本信息列表
\section*{}
\begin{itemize}[leftmargin=*]
    \item 学生姓名:张翔
    \item 学生学号:2023010909015
    \item 指导老师:王玉兰
    \item 实验地点:KA425
    \item 实验时间:星期四第9-10节课
    \item 选课序号:15
\end{itemize}

% 插入报告目录
\section*{报告目录}
\begin{enumerate}[label=\chinese*.]
    \item 实验课程名称:电路实验II
    \item 实验项目名称:\hrulefill % 使用横线作为占位符
    \item 实验目的:
        \begin{enumerate}[label=\arabic*.]
            \item 目的
        \end{enumerate}
    \item 设计任务与要求:
        \begin{enumerate}[label=\arabic*.]
            \item 目的
        \end{enumerate}
    \item 实验原理与方案设计:
    \item 实验内容、测试数据以及结论:
\end{enumerate}

% 将报告评分置于右下角
\noindent\vfill % 将内容推到页面底部
\hfill % 将内容推到页面右侧
\textbf{报告评分:}\hrulefill % 加粗报告评分并使用横线作为占位符

% 结束文档
\end{document}
